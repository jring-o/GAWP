\section{Financial}

\label{sec:finance}

A blockchain based financial system generated by the open and accessible analyzation of data would help redefine society’s value structures while simultaneously advancing scientific, medicinal, mathematical, and social practices.  The decentralized nature of blockchain technology paired with the rapid spread of processing power to individuals creates an ideal environment in which everyone has the opportunity to participate in the generation of this system’s currency.  Such a system would ultimately prioritize the pragmatic and democratized advancement of data based research in opposition to a system where advancements are chosen by the financial and social elite.

\subsection{Value}

Any contemporary blockchain holds value with relation to three major sources: \\

\begin{compactenum}
	\item Exchange: Its ability to rapidly and inexpensively exchange information through a decentralized and immutable mechanism.  \\
	\item Information: The information contained in the blockchain. \\
	\item Currency Generation: The generation and distribution mechanisms of the blockchain’s currency. \\
\end{compactenum}

Gridcoin is no exception. \\

\begin{compactenum}
	\item Exchange
\\ \\
In its current state, Gridcoin does not seek to revolutionize the way information is exchanged between participants of the blockchain.  The process is largely based off of Bitcoin protocols which have been comprehensively tested and secured.  As such, its value of exchange is directly related to the Bitcoin blockchain.
\\
	\item Information
\\ \\
A second point of value of the Gridcoin blockchain and network is through the value society places on information in the blocks and superblocks of the blockchain.  This perceived value is translated into exchangeable value through the currency generated by the blockchain, GRC.
\\ \\
A Gridcoin block contains information pertaining to the transfer of information between users participating in the Gridcoin blockchain.  In addition to these standard blocks, Gridcoin has developed a statistics collection tool which compiles individual user statistics into what we call a  Superblock.  Superblocks maintain all the qualities of standard blocks including immutability based on consensus.  Currently, Superblocks contain information related to Credits earned on the BOINC computing management platform.  Credits are information meant to represent the value given to completed Work Units.  Work Units are sets of data analyzed by processing power.  Each Superblock, it would follow, contains unique user information regarding the processing power that user contributes to BOINC projects.  Superblocks have the potential to collect statistics from a theoretically infinite array of sources.
\\
	\item Currency Generation
\\ \\
Gridcoin’s protocols generate GRC through two mechanisms.  First, GRC is generated and distributed to users for participating in the Proof of Stake security of the Gridcoin blockchain.  This is no different than many Proof of Stake blockchains and presents a similar point of value that relates to the value placed on the security of the blockchain.  The second GRC generation mechanism is unique to Gridcoin.  It uses the information compiled in superblocks.  GRC is generated and distributed based on a user’s statistics contained in Gridcoin superblocks.  This generation mechanism is as automated and transparent as other blockchain currency generation mechanisms.

\end{compactenum}

\subsection{Conclusion}

Just as a physical resource holds an exchange value equal to its perceived worth in a society, a blockchain’s generated currency, such as GRC, holds an exchange value equal to a society's perceived worth of the data in the blockchain combined with the currency’s generation protocol and the utility provided by the blockchain protocols.
